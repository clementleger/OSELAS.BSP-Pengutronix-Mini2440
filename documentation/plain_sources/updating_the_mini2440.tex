%
% Charset: utf8
% File type: Plain LATEX
% Content: This chapter describes the updating of the Mini2440.
%
% Copyright Jürgen Beisert <jbe@pengutronix.de>, 2011
%
% This work is licensed under the Creative Commons Attribution 3.0 Unported License.
% To view a copy of this license, visit:
%           http://creativecommons.org/licenses/by/3.0/
%
% Refer the file CREDITS for all people working on this document.
%
% This file content will be part of the "OSELAS.BSP-Pengutronix-Mini2440-Quickstart.pdf"
%
% Note: This document uses some externaly defined LATEX commands. If you try to
% run LATEX only on this file it will fail due to the absense of these commands
% All these commands are starting with 'ptxdist'.
%

\chapter{Updating the Mini2440}	\label{sec:updating}

At any time it's possible to update any of the software components
running on the Mini2440.

\begin{itemize}
	\item the bootloader Barebox
	\item Barebox's persistent environment
	\item the Linux kernel
	\item the kernel's root filesystem
\end{itemize}

This chapter explains how to update these parts for the various boot scenarious.
Note: This chapter is not finished yet.

\section{NOR flash memory case}

Not yet supported.

\section{NAND flash memory case}

If the Mini2440 uses the NAND flash memory to boot standalone, we can update
its components in the following way:

\subsection{Updating the Bootloader}

Most of the time there is no further need to re-flash the bootloader and its
persistent environment. After it was setup once, it does its work "in the
background". But nevertheless there could be the need to update the bootloader
due to feature additions or bug fixes. If the current Barebox bootloader is
still working, its replacement can be done by using the existing bootloader.
This assumes that the network is still functioning. In this case, a simple

\begin{ptxshell}[escapechar=|]{^}
|\$| cp |\ptxdistPlatformDir |images/barebox-image /tftpboot/barebox-mini2440
\end{ptxshell}

provides the updated bootloader binary via TFTP and a

\begin{ptxshell}[escapechar=|]{^}
mini2440:/ update -t barebox -d nand
\end{ptxshell}

will do the job at the target side. After starting this command, do not disturb!
This is a critical update process. Because, for a short period of time the
NAND flash is erased, with no bootloader present. But don't panic: Unless a
power fail or a target reset happens, this command can be repeated if the
first run failed.

\subsection{Updating the Persistent Environment}

% I'm unsure how useful it is: Assuming the current environment is broken, I
% guess also the network setup is broken. So, loading via TFTP maybe is
% impossible.
% Idea:
% Maybe we should think about disasters a user can happen, and how to bring
% back life into her/his Mini2440 in this case

Updating the persistent environment is also possible. A simple

\begin{ptxshell}[escapechar=|]{^}
|\$| cp |\ptxdistPlatformDir |images/barebox-default-environment /tftpboot/barebox-default-environment-mini2440
\end{ptxshell}

provides the updated environment via TFTP and a

\begin{ptxshell}[escapechar=|]{^}
mini2440:/ update -t bareboxenv -d nand
\end{ptxshell}

will do the job. Note: This new persistent environment will be used at the next
system start.

If the persistent environment is broken, there is a second method to restore a
working environment: that is, using the compiled in default environment which
comes with Barebox.
To force the usage of the compiled in default environment, just erase the
current one in the NAND flash memory and reset the target (or run the
\texttt{reset} command).

\begin{ptxshell}[escapechar=|]{^}
mini2440:/ erase /dev/bareboxenv.bb
mini2440:/ reset
\end{ptxshell}

Now, Barebox will stumble about the empty partition and then fall back to its
compiled-in environment version. This can now be changed by editing the files
in \texttt{env/} and then saved back to the NAND flash memory with the command

\begin{ptxshell}[escapechar=|]{^}
mini2440:/ saveenv
\end{ptxshell}

\subsection{Updating the Linux Kernel}

Changing the Linux kernel configuration can be quite dynamic, especially while
the developer is trying different kernel configurations. Updating this part
happens in the same way like the other parts. Providing the Linux kernel via
TFTP:

\begin{ptxshell}[escapechar=|]{^}
|\$| cp |\ptxdistPlatformDir |images/linuximage /tftpboot/uImage-mini2440
\end{ptxshell}

and running the update script at the target's side:

\begin{ptxshell}[escapechar=|]{^}
mini2440:/ update -t kernel -d nand
\end{ptxshell}

\subsection{Updating the Root Filesystem}

And last, but not least, updating the root filesystem. Same procedure:

\begin{ptxshell}[escapechar=|]{^}
|\$| cp |\ptxdistPlatformDir |images/root.jffs2 /tftpboot/root-mini2440.jffs2
\end{ptxshell}

and running the update script at the target's side:

\begin{ptxshell}[escapechar=|]{^}
mini2440:/ update -t rootfs -d nand
\end{ptxshell}

\section{SD/MMC card memory case}

We assume here the partitioning is done as mentioned in section
\ref{sec:boot_sdcard}: two partitions on the SD/MMC card, first contains the
kernel image and no filesystem, second contains the root filesystem content
with a filesystem we like. We also assume here, updating the card content is
not done at the target's side, it is still done at our host. This means, we are
using some kind of USB based card reader and these kind of devices show us the
attached cards as SCSI devices.

\subsection{Updating the Linux Kernel}

Changing the kernel is rather simple:

\begin{ptxshell}[escapechar=|]{^}
|\$| sudo cp |\ptxdistPlatformDir |images/linuximage /dev/sdd1
\end{ptxshell}

Note: Replace the \texttt{/dev/sdd1} here with the device node name the card
reader is visible in your system.

\subsection{Updating the Root Filesystem}

Changing the root filesystem content mostly means we must remove its old content.
This can be done quickly by re-formatting it.

\begin{ptxshell}[escapechar=|]{^}
|\$| sudo mkfs.ext2 /dev/sdd2
|\$| sudo mount /dev/sdd2 /mnt
|\$| sudo tar -xvzf |\ptxdistPlatformDir |images/root.tgz -C /mnt
|\$| sudo umount /mnt
\end{ptxshell}

\section{Network memory case}

As this kind of memory is on our host, everything we change will be also
changed instantly at the Mini2440's side (as long we are talking about the
kernel and the root filesystem).

% the following text should occur somewhere else, where it is more useful (?)
By the way: the \texttt{update} command is not a real command built into
Barebox. Its a simple shell script coming from the persistent environment.
If one has different update scenarios she/he can change or adapt this script.
Changing this behaviour can be done without touching Barebox's source code.

%
% Ideas:
% - Add info about how to test the Barebox written to the NAND (if it was
%   successful)
% - Add info on how to check the downloaded file is the correct one
% - maybe some comment about <erase /dev/nand0> and the use of <reset>
% - if you don't change anything using the edit command do you need to <saveenv>?  With
%   u-boot, most tutorials suggested a <saveenv>
% - What is dangerous about a <printenv>?
