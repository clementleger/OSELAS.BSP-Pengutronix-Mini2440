%
% Charset: utf8
% Content type: Plain LATEX
% Content: This chapter give some Mini2440 specific hints how to use the
%          hardware provided by the Mini2440.
%
% Copyright Jürgen Beisert <jbe@pengutronix.de>, 2011
%
% This work is licensed under the Creative Commons Attribution 3.0 Unported License.
% To view a copy of this license, visit:
%           http://creativecommons.org/licenses/by/3.0/
% or send a letter to:
% Creative Commons, 444 Castro Street, Suite 900, Mountain View, California, 94041, USA.
%
% Refer the file CREDITS for all people working on this document.
% %
% This file content will be part of the "OSELAS.BSP-Pengutronix-Mini2440-Quickstart.pdf"
%
% Note: This document uses some externaly defined LATEX commands. If you try to
% run LATEX only on this file it will fail due to the absense of these commands.
% All these commands are starting with 'ptxdist'.
%

\newcommand{\perCpuName}{S3C2440}

\section{Available Kernel Releases}	\label{sec:kernelreleases}

The predifined Mini2440 platform configuration always uses the latest Linux
kernel release. If users want to stay with an older Linux kernel release,
they are also available. Here is a list of currently available Linux kernel
releases in the \ptxdistBSPName{}:

\begin{itemize}
	\item 2.6.39 (default)
	\item 2.6.38.7
\end{itemize}

If you want to build the BSP with an non-default kernel release, just run
\texttt{ptxdist platformconfig} and change the kernel setting prior to building.

% ----------------------------------------------------------------------------

\newcommand{\perDisplayName}{LCDN3502-23B}
\newcommand{\perDisplayRes}{240x320}

\section{Framebuffer}					\label{sec:fb}
% --------------------

This driver gains access to the display via device node \texttt{/dev/fb0}.
For this BSP the \perDisplayName{} display with a resolution of
\perDisplayRes{} is supported.

A simple test of this feature can be run with:

\begin{ptxshell}{|}
# |fbtest|
\end{ptxshell}

This will show various pictures on the display.

You can check your framebuffer resolution with the command
\begin{ptxshell}{|}
# |fbset|
\end{ptxshell}

NOTE: \texttt{fbset} cannot be used to change display resolution or color depth.
Depending on the framebuffer device different kernel command line may be needed
to do this. Please refer to your display driver manual for details.

% ----------------------------------------------------------------------------

\newcommand{\perGpioChip}{192}
\newcommand{\perGpioNumber}{193}

\section{GPIO}					\label{sec:GPIO}
% ------------

Like most modern System-on-Chip CPUs, the \perCpuName{} has numerous GPIO pins.
Some of them are unaccessible for the userspace as Linux drivers use them
internally. Others are also used by drivers but are exposed to userspace via
sysfs. Finally, the remaining GPIOs can be requested for custom use by
userspace, also via sysfs.

Refer to the in-kernel documentation \texttt{Documentation/gpio.txt} for
complete details how to use the sysfs-interface for manually exporting GPIOs.

% ----------------------------------------------------------------------------

\subsection{GPIO Usage Example}		\label{sec:gpioexample}

When generic architecture GPIO support is enabled in the kernel, some new
entries appear in sysfs. Everything is controlled via read and writable
files to generate events on the digital lines.

We find all the control files in \texttt{/sys/class/gpio}. In that path, there
a number of \texttt{gpiochipXXX} entries, with \texttt{XXX} being a decimal
number. Each of these folders provide information about a single gpio
controller registered on the Mini2440 board, for example with
\texttt{gpiochip\perGpioChip}:

\begin{ptxshell}[escapechar=^]{|}
# |ls /sys/class/gpio/gpiochip^\perGpioChip^|
base       label      ngpio      subsystem  uevent
\end{ptxshell}

The entry \texttt{base} contains information about the base GPIO number and
\texttt{ngpio} contains all GPIOs provided by this GPIO controller.\\
We use \texttt{GPIO\perGpioNumber{}} as an example to show the usage of a
single GPIO pin.

\begin{ptxshell}[escapechar=^]{|}
# |echo ^\ptxshellcmd\perGpioNumber^ > /sys/class/gpio/export|
\end{ptxshell}

This way we export \texttt{gpio\perGpioNumber{}} for userspace usage. If the
export was succesful, we will find a new directory named
\texttt{/sys/class/gpio/gpio\perGpioNumber{}} afterwards. Within this directory
we will be able to find the entries to access the functions of this gpio.
If we wish to set the direction and initial level of the GPIO, we can use the
command:

\begin{ptxshell}[escapechar=^]{|}
# |echo high > /sys/class/gpio^\perGpioNumber^/direction|
\end{ptxshell}

This way we export \texttt{GPIO\perGpioNumber{}} for userspace usage and define
our GPIO's direction attribute to an output with initially high level. We can
change the value or direction of this GPIO by using the entries
\texttt{direction} or \texttt{value}.

Note: This method is not very fast, so for quickly changing GPIOs it is still
necessary to write a kernel driver. The method shown works well, for example to
influence an LED directly from userspace.

To unexport an already exported GPIO, write the corresponding gpio-number into
\texttt{/sys/class/gpio/export}.

\begin{ptxshell}[escapechar=^]{|}
# |echo ^\perGpioNumber^ > /sys/class/gpio/unexport|
\end{ptxshell}

Now the directory \texttt{/sys/class/gpio/gpio\perGpioNumber} will disapear.

Note: The GPIO\perGpioNumber{} is available at connector 4, pin 17 for
measurement.

% ----------------------------------------------------------------------------

\section{I\texttwosuperior C Master}			\label{sec:I2C}
% -------------------------------------

The \perCpuName{} processor based Mini2440 supports a dedicated
I\texttwosuperior C controller onchip. The kernel supports this controller as a
master controller.

Additional I\texttwosuperior C device drivers can use the standard
I\texttwosuperior C device API to gain access to their devices through this
master controller.
For further information about the I\texttwosuperior C framework see
\texttt{Documentation/i2c} in the kernel source tree.

% ----------------------------------------------------------------------------

\newcommand{\perEepromName}{AT24c08}
\newcommand{\perEepromSize}{1024 bytes}
\newcommand{\perEepromPath}{0-0050}

\subsection{I\texttwosuperior C Device \perEepromName {}}	\label{sec:EEPROM}
% ------------------------------------------------------------

This device is a \perEepromSize{} non-volatile memory for general purpose usage.

This type of memory is accessible through the sysfs filesystem. To read the
EEPROM content simply \texttt{open()} the entry
\texttt{/sys/bus/i2c/devices/\perEepromPath{}/eeprom} and use
\texttt{fseek()} and \texttt{read()} to get the values.

% ----------------------------------------------------------------------------

\newcommand{\perLedName}{led1}
\newcommand{\perAvailableTriggers}{[none] nand-disk mmc0 timer backlight}

\newcommand{\LedPath}{/sys/class/leds/\perLedName\#}

\section{LEDs}				\label{sec:LED}
% ------------

The LEDs on the Mini2440 can be controlled via the LED-subsystem of the Linux
kernel. It provides methods for switching them on and off as well as using
so-called triggers.  For example, you could trigger the LED using a timer.
That enables us to make it blink with any frequency we want.

All LEDs can be found in the directory \texttt{/sys/class/leds}. Each one has
its own subdirectory. We will use \texttt{\perLedName} for the following
examples.

Per directory, you have a file named \texttt{brightness} which can be read and
written with a decimal value between 0 and 255. The first one means LED off,
the latter maximum brightness. Inbetween values scale the brightness if the LED
supports that. If not, non-zero means just LED on.

% not aware that brightness is dependent on support from the LED itself.  Isn't
% <brightness> just affecting the duty cycle of a PWM? Are there some special L
% support this type of operation?

\begin{ptxshell}[escapechar=|]{^}
|\LedPath| echo 255 > brightness; # LED on
|\LedPath| echo 128 > brightness; # LED at 50% (if supported)
|\LedPath| echo 0 > brightness; # LED off
\end{ptxshell}

LEDs can be connected to triggers. A list of available triggers we can get from the
\texttt{trigger} entry

\begin{ptxshell}[escapechar=|]{^}
|\LedPath| cat trigger
|\perAvailableTriggers{}|
\end{ptxshell}

The embraced entry is the currectly connected trigger to this LED.

To change the trigger source to the \textit{timer}, just run a:

\begin{ptxshell}[escapechar=|]{^}
|\LedPath| echo timer > trigger
\end{ptxshell}

If the timer-trigger is activated you should see two additional files in the
current directory, namely \texttt{delay\_on} and \texttt{delay\_off}. You can
read and write decimal values there, which will set the corresponding delay in
milliseconds. As an example:

\begin{ptxshell}[escapechar=|]{^}
|\LedPath| echo 250 > delay_on
|\LedPath| echo 750 > delay_off
\end{ptxshell}

will blink the LED being on for 250ms and off for 750 ms.

Replace \texttt{timer} with \texttt{none} to disable the trigger again. Or
select a different one from the list read from the \texttt{trigger} entry.

Refer to \texttt{Documentation/leds-class.txt} in-kernel documentation for
further details about this subsystem.

% ----------------------------------------------------------------------------

\section{MMC/SD Card}					\label{sec:SDC}
%-----------------------

The Mini2440 supports \textit{Secure Digital Cards} and
\textit{Multi Media Cards} to be used as general purpose blockdevices. These
devices can be used in the same way as any other blockdevice.

\begin{important}
These kind of devices are hot pluggable, so you must pay attention not to
unplug the device while its still mounted. This may result in data loss.
\end{important}

After inserting an MMC/SD card, the kernel will generate new device nodes in
\texttt{dev/}. The full device can be reached via its \texttt{/dev/mmcblk0}
device node, MMC/SD card partitions will occur in the following way:

\begin{ptxshell}{|}
/dev/mmcblk0p|Y|
\end{ptxshell}
\texttt{\textbf{Y}} counts as the partition number starting from 1 to the max
count of partitions on this device.

Note: These partition device nodes will only occur if the card contains a
valid partition table ("harddisk" like handling). If it does not contain one,
the whole device can be used for a filesystem ("floppy" like handling). In
this case \texttt{/dev/mmcblk0} must be used for formatting and mounting.

The partitions can be formatted with any kind of filesystem and also handled in
a standard manner, e.g. the \texttt{mount} and \texttt{umount} command work as
expected.

% ----------------------------------------------------------------------------

\newcommand{\perNetworkChip}{ DM9000}
\newcommand{\perNetworkInterface}{eth0}

\section{Network}				\label{sec:NET}
% ------------------

The Mini2440 module has a\perNetworkChip{} ethernet chip onboard,
which is being used to provide the \texttt{\perNetworkInterface{}} network
interface. The interface offers a standard Linux network port which can be
programmed using the BSD socket interface.

% ----------------------------------------------------------------------------

\section{SPI Master}				\label{sec:SPI}
% ---------------------

The Mini2440 board supports an SPI bus, based on the \perCpuName 's
integrated SPI controller. It is connected to the onboard devices using the
standard kernel method, so all methods described here are not special to the
Mini2440.

Connected devices can be found in the sysfs at the path
\texttt{/sys/bus/spi/devices}. It depends on the corresponding SPI slave device
driver providing access to the SPI slave device through this way (sysfs),
or any different kind of API.

\begin{important}
Currently no SPI slave devices are registered, so the
\texttt{/sys/bus/spi/devices} directory is empty.
\end{important}

%
% Idea: Describe how to add an existing SPI device driver to the platform
%

% ----------------------------------------------------------------------------

\input{Customer/peripheral-library/touch.tex}		\label{sec:TOUCH}

To see the exact events the touch generates, we can also use the
\texttt{evtest} tool.

\begin{ptxshell}[escapechar=|]{^}
# evtest /dev/input/event1
Input driver version is 1.0.1
Input device ID: bus 0x19 vendor 0xdead product 0xbeef version 0x102
Input device name: "S3C24XX TouchScreen"
Supported events:
  Event type 0 (Sync)
  Event type 1 (Key)
    Event code 330 (Touch)
  Event type 3 (Absolute)
    Event code 0 (X)
     Value      0
     Min        0
     Max     1023
  Event code 1 (Y)
     Value      0
     Min        0
     Max     1023
Testing ... (interrupt to exit)
\end{ptxshell}

Whenever we touch the screen this tool lists the values the driver reports.

% ----------------------------------------------------------------------------

\section{USB Host Controller Unit}			\label{sec:USBHOST}
% -----------------------------------

The Mini2440 supports a standard OHCI Rev. 1.0a compliant host controller
onboard for low and full speed connections.

%
% Idea: Describe how to add support for various USB devices
%

% ----------------------------------------------------------------------------

\newcommand{\perWatchdogTimeout}{15}

\section{Watchdog}			\label{sec:WD}
% -------------------

The internal watchdog will be activated when an application opens the device
\texttt{/dev/watchdog}. Default timeout is \perWatchdogTimeout{} seconds.
An application must periodically write to this device. It does not matter what
is written. Just the interval between these writes should not exceed the
timeout value, otherwise the system will be reset.

For testing the hardware, there is also a shell command which can do the
triggering:

\begin{ptxshell}{|}
# watchdog -t <trigger-time-in-seconds> /dev/watchdog
\end{ptxshell}

This command is part of the busybox shell environment. Keep in mind, that
it should only be used for testing. If the watchdog gets fed by it, a crash of
the real application will go unnoticed.

For the Mini2440 the default 60 seconds interval period the
tool is using is too long. The driver for the \perCpuName{} only supports up to
a 40 seconds interval. So, the additional parameter \texttt{-T 40} must be
given.

%----------------------------------------------------------------------------

%
% What is still missing?
%
% - how to use the 6 keys on the Mini2440
% - how to work with the ADCs (and with W1 to see how it works)
% - how to use the Mini2440 as a gadget (printer, serial or ...)
% - on the FriendlyARM someone asks how to implement a W1 controller
% - description whats to be done in case of SDRAM errors (this bootloader
%   sets up a faster speed than the vivi does)
% - how to play sounds
% - what is to be done, if one is using a different LC display than my LCDN3502
%
% Ideas:
% - how to speed up booting
% - boot splash
% - how to bring in Qt and an example application
% - how to bring up a local terminal and accept USB keyboard input

%----------------------------------------------------------------------------

\section{Get the latest BSP Release for the Mini2440}

Information and the latest release of the Mini2440 BSP can be found on
our website at:

\url{http://www.oselas.org/oselas/bsp/index\_en.html}

%----------------------------------------------------------------------------

\section{Be Part of the Mini2440 BSP Development}

If you want to use the latest and greatest board support package for the
Mini2440 you can use the git repository as your working source, instead
of a release archive.

The git repository can be found here:

\begin{center}
\texttt{http://git-public.pengutronix.de/git-public/OSELAS.BSP-Pengutronix-Mini2440.git}
\end{center}

If you want to contribute to this project by sending patches, these patches
should always be based on the \textbf{master} branch of this repository.
