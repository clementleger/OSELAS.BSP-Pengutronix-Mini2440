%
% Charset: utf8
% Content type: Plain LATEX
% Content: This chapter gives the user some knowledge about using NAND memory
%          in her/his application. There are some disadvantages everybody
%          should know.
%
% Copyright Jürgen Beisert <jbe@pengutronix.de>, 2011
% Copyright Jürgen Borleis <jbe@pengutronix.de>, 2014
%
% This work is licensed under the Creative Commons Attribution 3.0 Unported License.
% To view a copy of this license, visit:
%           http://creativecommons.org/licenses/by/3.0/
% or send a letter to:
% Creative Commons, 444 Castro Street, Suite 900, Mountain View, California, 94041, USA.
%
% Refer the file CREDITS for all people working on this document.
% %
% This file content will be part of the "OSELAS.BSP-Pengutronix-Mini2440-Quickstart.pdf"
%

% ----------------------------------------------------------------------------
\chapter{You have been warned}
% ----------------------------------------------------------------------------

The Barebox bootloader and the Linux kernel contained in this board support
package will modify your NAND memory. This is important to know if you want to
keep a way back to the previous usage. At least the bad block marker maybe
lost if you try to switch back to the old behaviour.

If you already used another recent kernel on your Mini2440, you can ignore this
warning.

A word about using NAND memory for the bootloader and the filesystem:

NAND memory can be forgetful. That is why some kind of redundancy information
is always required. This board support package uses ECC (error-correcting code)
checksums as redundancy information when the bootloader and the Linux kernel
are up and running.

This kind of redundancy information can repair one bit errors and detect two
bit errors in a page of data. Its very important to use ECC at least for the
bootloader to ensure to bring up the Mini2440 successfully. But its currently
not done in the bootloader while bootstrapping. So, there is still a risk for
long term use to fail booting the Mini2440 from NAND. In this case the
bootloader must be re-written making the Mini2440 booting again from NAND.

In one of the next releases, ECC check and correction will be done while
bootstrapping as well, to make the system more reliable for long term use.
But then one question will be still open: Does the hardware of the S3C2440 CPU
ECC check and correction for the very first page? I guess no, because the
hardware has no idea, where the ECC checksum is stored. So, maybe there is no
100 \% reliable solution for long term users.
