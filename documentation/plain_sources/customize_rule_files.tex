%
% Charset: utf8
% Content type: Plain LATEX
% Content: This chapter describes how the customize the Mini2440 Board Support
%          Package with the help of PTXdist.
%
% Copyright Dave Festing <dave_festing@hotmail.com>, 2012
%
% This work is licensed under the Creative Commons Attribution 3.0 Unported License.
% To view a copy of this license, visit:
%           http://creativecommons.org/licenses/by/3.0/
% or send a letter to:
% Creative Commons, 444 Castro Street, Suite 900, Mountain View, California, 94041, USA.
%
% Refer the file CREDITS for all people working on this document.
%
% This file content will be part of the "OSELAS.BSP-Pengutronix-Mini2440-Quickstart.pdf"
%
% Note: This document uses some externaly defined LATEX commands. If you try to
% run LATEX only on this file it will fail due to the absense of these commands
% All these new commands are starting with 'ptxdist'.
%

\section{HowTo customise your root filesystem and update PTXdist with it}

For the past year when I wanted to include a binary, such as msmtp, or add a
config file, such as msmtprc, I just added these files to
\texttt{\~/my\_PTXdist/platform-mini2440/root/} folder. The problem here is that
anytime I did a \texttt{ptxdist clean root} I would lose these files.
Fortunately, there is a proper way to do this.

The 'secret' is revealed by looking into \texttt{/usr/local/lib/ptxdist-version/rules}
and specifically at \texttt{rootfs.in} and \texttt{rootfs.make}. These two
files give you the format to make your own \texttt{daves\_rootfs.in} and
\texttt{daves\_rootfs.make} files which will live in \texttt{\~/your\_PTXdist/rules}
folder.

For example here is my added \texttt{daves\_rootfs.in} file:

\begin{ptxshell}[escapechar=|]{^}
## SECTION=rootfs

menuconfig DAVES_ROOTFS
	bool
	prompt "files in dave's rootfs        "
	help
	  Dave's special files to make his target work

if DAVES_ROOTFS

config DAVES_ROOTFS_MSMTPRC_DAVE
	bool
	prompt "install /etc/msmtprc-dave"
	help
	  If enabled, it installs the "./projectroot/msmtprc" file

config DAVES_ROOTFS_MSMTPRC_CHARLES
	bool
	prompt "install /etc/msmtprc-charles"
	help
	  If enabled, it installs the "./projectroot/msmtprc-charles" file

config DAVES_ROOTFS_LIB_UDEV_RULESD_TTYACM0
	bool
	prompt "install /lib/udev/rules.d/50-udev-default.rules"
	help
	  If enabled, it installs the "./projectroot/lib/udev/rules.d/50-udev-default.rules"
	  file if present, else a generic one from the ptxdist directory.

config DAVES_ROOTFS_TEST_PHP
	bool
	prompt "install /home/test.php"
	help
	  If enabled, it installs the "./projectroot/home/test.php" file if present.

endif
\end{ptxshell}

When you do \texttt{ptxdist menuconfig} and look in \texttt{Root Filesystem --->}
you will see a new heading called:

\begin{ptxshell}[escapechar=|]{^}
 [ ] files in dave's rootfs ---->
\end{ptxshell}

Say \texttt{[y]} and in there you can tick your special files that you want
to include.

Next you have to make your own \texttt{daves\_rootfs.make} file. Here's an example:

\begin{ptxshell}[escapechar=|]{^}
# -*-makefile-*-
#
# Copyright (C) 2012 Dave Festing <dave_festing@hotmail.com>
#
# See CREDITS for details about who has contributed to this project.
#
# For further information about the PTXdist project and license conditions
# see the README file.
#

#
# We provide this package
#
PACKAGES-|\$|(PTXCONF_DAVES_ROOTFS) += daves_rootfs

# dummy to make ipkg happy
DAVES_ROOTFS_VERSION	:= 1.0.0

# ----------------------------------------------------------------------------
# Target-Install
# ----------------------------------------------------------------------------

|\$|(STATEDIR)/daves_rootfs.targetinstall:
	@|\$|(call targetinfo)

	@|\$|(call install_init,  daves_rootfs)
	@|\$|(call install_fixup, daves_rootfs,PRIORITY,optional)
	@|\$|(call install_fixup, daves_rootfs,SECTION,base)
	@|\$|(call install_fixup, daves_rootfs,AUTHOR,"Dave Festing <dave_festing@hotmail.com>")
	@|\$|(call install_fixup, daves_rootfs,DESCRIPTION,missing)

# my install_alternative files
ifdef PTXCONF_DAVES_ROOTFS_MSMTPRC_DAVE
	@|\$|(call install_alternative, daves_rootfs, 0, 0, 0644, /etc/msmtprc-dave)
endif
ifdef PTXCONF_DAVES_ROOTFS_MSMTPRC_CHARLES
	@|\$|(call install_alternative, daves_rootfs, 0, 0, 0644, /etc/msmtprc-charles)
endif
ifdef PTXCONF_DAVES_ROOTFS_LIB_UDEV_RULESD_TTYACM0
	@|\$|(call install_alternative, daves_rootfs, 0, 0, 0644, /lib/udev/rules.d/50-udev-default.rules)
endif
ifdef PTXCONF_DAVES_ROOTFS_TEST_PHP
	@|\$|(call install_alternative, daves_rootfs, 0, 0, 0755, /home/test.php)
endif

	@|\$|(call install_finish, daves_rootfs)
	@|\$|(call touch)

# vim: syntax=make
\end{ptxshell}

Some comments about this file:

\begin{ptxshell}[escapechar=|]{^}
ifdef PTXCONF_DAVES_ROOTFS_LIB_UDEV_RULESD_TTYACM0
	@|\$|(call install_alternative, daves_rootfs, 0, 0, 0644, /lib/udev/rules.d/50-udev-default.rules)
endif
\end{ptxshell}

This is a special file to modify \texttt{50-udev-default.rules} so that I could
plug in a USB dev board that thinks it is a modem. See \textit{Teensy and Atmega32U4}
dev board from Adafruit for details. \texttt{install\_alternative} is the keyword
that tells \ptxdist{}, in this case, to go looking for a file in
\texttt{\~/your\_PTXdist/projectroot/lib/udev/rules.d} folder and over-write the
existing \texttt{50-udev-default.rules} file from the \texttt{udev} package
with this special one. And give it the permissions 0644.

\section{Updating to a new PTXdist version}

Copy all the files in your old \texttt{\~/your\_PTXdist/projectroot/} folder to
somewhere for backup and then into your new \texttt{\~/your\_PTXdist/projectroot/}
folder. Likewise, copy the \texttt{daves\_rootfs.in} and \texttt{daves\_rootfs.make}
files somewhere for backup and then into your new \texttt{\~/your\_PTXdist/rules}
folder.

I believe that should just work. Updating \ptxdist{} versions may bring up
changes that need sorting out. Certainly, updating kernel versions need more
changes, but I guess that Pengutronix take care of these problems when they
update their BSPs.

Troubleshooting:

One issue that will really upset \ptxdist{} is if the line endings are not
Linux line endings. I still haven't worked out how these line endings ever got
changed in the process of using the examples provided by Juergen at Pengutronix,
but here is the command that made them work:

\begin{ptxshell}[escapechar=|]{^}
sed 's/.|\$|//' daves_new.in > daves_rootfs.in
\end{ptxshell}

Of course change all files names to something more appropriate to you!
