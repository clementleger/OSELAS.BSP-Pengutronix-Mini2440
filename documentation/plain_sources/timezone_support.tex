%
% Charset: utf8
% Content type: Plain LATEX
% Content: This chapter give some information how to setup a correct timezone
%          information with the help of PTXdist
%
% Copyright Jürgen Beisert <jbe@pengutronix.de>, 2012 and with the help of
%           Dave Festing <dave_festing@hotmail.com>, 2012
%
% This work is licensed under the Creative Commons Attribution 3.0 Unported License.
% To view a copy of this license, visit:
%           http://creativecommons.org/licenses/by/3.0/
% or send a letter to:
% Creative Commons, 444 Castro Street, Suite 900, Mountain View, California, 94041, USA.
%
% Refer the file CREDITS for all people working on this document.
%
% This file content will be part of the "OSELAS.BSP-Pengutronix-Mini2440-Quickstart.pdf"
%
% Note: This document uses some externaly defined LATEX commands. If you try to
% run LATEX only on this file it will fail due to the absense of these commands.
% All these commands are starting with 'ptx...'.
%
\section{Using time zone support on the Mini2440}	\label{sec:timezone}

The Mini2440 comes with a \textit{real time clock (RTC)} which keeps track of
the time even when the Mini2440 is switched off (assumed here is a working
battery). With the command \texttt{hwclock} we can set and get the current
time into or from the RTC. The kernel from the Mini2440 BSP is configured to
read the current time from the RTC at system startup to use it as its system
time. The \texttt{date} command shows this system time.

\begin{ptxshell}[escapechar=|]{^}
|\#| date
Sat Jul 28 13:16:08 UTC 2012
|\#| hwclock
Sat Jul 28 13:16:12 2012  0.000000 seconds
|\#| hwclock -u
Sat Jul 28 13:16:18 2012  0.000000 seconds
\end{ptxshell}

Both commands show the time for the \textit{Universal Time Coordinated (UTC)}
timezone. But most of the time this is not what we want. We want to have a
localized time. So, we have to give these commands some information about our
local timezone. \ptxdist{} supports the installation of timezone information
files. But this feature is not enabled by default, because we do not know where
all the users of the Mini2440 live. And not all users live in central
Europe...

We need to understand that Linux always uses UTC time for its system time.
There are various tools and library functions for handling
the time and only these tools and functions will change the time to the local
timezone on demand.
With this knowledge we must set the time in the RTC always in UTC. But we
shouldn't worry about that. There is no need to manually add or subtract the
difference from our local time to the UTC. When the local timezone information
is available in the system, all the tools do it automatically.

Let us assume we live in New Zealand (main islands) and so we are 12 hours in
advance of the UTC timezone. There are two ways available to get the correct
timezone information installed:

\begin{ptxshell}[escapechar=|]{^}
|\$| ptxdist menuconfig
  Core (libc, locales) --->
    <*> Timezone Files --->
      (NZ) Timezone for /etc/localtime and /etc/timezone
      [*]   Use local build timezone database
      [*]   NZ
\end{ptxshell}

Most tools and library functions use the file \texttt{/etc/localtime} to
get information about the local timezone. But the timezone information files
are stored to \texttt{/usr/share/zoneinfo}.\\
To keep the filesystem small the \texttt{/etc/localtime} file should be just a
softlink to the corresponding file in the \texttt{/usr/share/zoneinfo}
directory. That is why we need to setup the correct filename from the timezone
information file into the \texttt{Timezone for /etc/localtime and /etc/timezone}
menu line. Otherwise the created \texttt{/etc/localtime} softlink is broken as
it points to nothing.

After populating the root filesystem with these timzone extensions, the
\texttt{date} and \texttt{hwclock} commands act differently:

\begin{ptxshell}[escapechar=|]{^}
|\#| date
Sun Jul 29 01:49:47 NZST 2012
|\#| hwclock
Sat Jul 28 13:49:48 2012  0.000000 seconds
|\#| hwclock -u
Sun Jul 29 01:49:50 2012  0.000000 seconds
\end{ptxshell}

Different to the UTC example shown above, they now show localized times. The
\texttt{hwclock} command prints the RTC time 'as is' as it assumes the RTC
stored time is already localized. \texttt{hwclock -u} acts different, as it
assumes the RTC time is really UTC and also calculates the local time from the
read value.

If we want to use a more precise timezone (New Zealand has two of them), we
should enable the \texttt{Pacific} entry in the \ptxdist{} menu.

\begin{ptxshell}[escapechar=|]{^}
|\$| ptxdist menuconfig
  Core (libc, locales) --->
    <*> Timezone Files --->
      (Pacific/Auckland) Timezone for /etc/localtime and /etc/timezone
      [*]   Use local build timezone database
      [*]   Pacific
\end{ptxshell}

This one gives us the same result as the \texttt{NZ} example shown above, while
changing it to:

\begin{ptxshell}[escapechar=|]{^}
|\$| ptxdist menuconfig
  Core (libc, locales) --->
    <*> Timezone Files --->
      (Pacific/Chatham) Timezone for /etc/localtime and /etc/timezone
      [*]   Use local build timezone database
      [*]   Pacific
\end{ptxshell}

selects a timezone information for the \textit{Chatham Islands} which are 12
hours and 45 minutes in advance of UTC.

\begin{ptxshell}[escapechar=|]{^}
|\#| date
Sun Jul 29 02:55:36 CHAST 2012
|\#| hwclock
Sat Jul 28 14:10:39 2012  0.000000 seconds
|\#| hwclock -u
Sun Jul 29 02:55:40 2012  0.000000 seconds
\end{ptxshell}

But we must keep in mind: if we want to set the time of the RTC, we always
must set it in UTC. But if the local timezone information is present in the
Mini2440's root filesystem this is very easy. We just have to set the system
time with our current local time and then write the system time into the RTC
with a parameter that will really store it in UTC. Thanks to the timezone
information files, the \texttt{hwclock -uw} command will do it automatically
for us.

\begin{ptxshell}[escapechar=|]{^}
|\#| date "2012-07-28 14:35"
Sat Jul 28 14:35:00 CHAST 2012
|\#| hwclock -uw
\end{ptxshell}
