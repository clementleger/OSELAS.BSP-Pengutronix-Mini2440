%
% Charset: utf8
% Content type: Plain LATEX
% Content: This chapter give some hints about using a foreign toolchain to
%          build the "OSELAS.BSP-Pengutronix-Mini2440" board support package
%
% Copyright Jürgen Beisert <jbe@pengutronix.de>, 2012
%
% This work is licensed under the Creative Commons Attribution 3.0 Unported License.
% To view a copy of this license, visit:
%           http://creativecommons.org/licenses/by/3.0/
% or send a letter to:
% Creative Commons, 444 Castro Street, Suite 900, Mountain View, California, 94041, USA.
%
% Refer the file CREDITS for all people working on this document.
% %
% This file content will be part of the "OSELAS.BSP-Pengutronix-Mini2440-Quickstart.pdf"
%
% Note: This document uses some externaly defined LATEX commands. If you try to
% run LATEX only on this file it will fail due to the absense of these commands.
% All these commands are starting with 'ptxdist'.
%

\section{Using a Foreign Toolchain}		\label{sec:foreigntoolchain}

It is possible to use a different toolchain than the recommended OSELAS.Toolchain.
But to make sure the result will work on the Mini2440, some preconditions must
be met.

The Mini2440 uses a Samsung S3C2440 processor. This processor comes with a so
called \textit{ARMv4 core} (also called \textit{the architecture}). This means
some internal features this processor supports and - much more important - the
command set it understands.

So, an important precondition to use a foreign toolchain is, it must generate
code only for this \textit{ARMv4 core}. There are always two ways to ensure the
compiler generates code that matches the \textit{architecture}:

\begin{itemize}
	\item the default settings are matching the \textit{ARMv4 core architecure}
	\item a compile parameter temporarely switches to this \textit{ARMv4 core architecure}
\end{itemize}

The first way was set, when the toolchain was built. In the case of our
OSELAS.Toolchain we configure the compiler with:

\texttt{-{}-with-arch=armv4t}

which results into the default compiler option for code generation if not
otherwise specified later on.

The second way can be archived by using the compiler parameter
\texttt{-march=armv4t}. This will switch the code generation to the
\textit{ARMv4 core architecure} for the run of the compiler.

\subsection{Discovering Toolchain's Compiler Defaults}

Two ways do exist to get the default settings out of the toolchain's compiler.

\subsubsection{By Preprocessor Macros}

Run the following command and examine the result:

\begin{ptxshell}[escapechar=~]{^}
~\$~ echo "" | arm-linux-gcc -E -dM - | grep __ARM_ARCH_
#define __ARM_ARCH_4T__ 1
\end{ptxshell}

If the listed \textbf{\_\_ARM\_ARCH\_} macro contains higher architecture numbers
than the shown \textbf{4} this toolchain cannot be used for the Mini2440.

Currently known \textbf{not} working architectures are:

\begin{itemize}
	\item \_\_ARM\_ARCH\_5\_\_
	\item \_\_ARM\_ARCH\_5E\_\_
	\item \_\_ARM\_ARCH\_5T\_\_
	\item \_\_ARM\_ARCH\_5TE\_\_
	\item \_\_ARM\_ARCH\_5TEJ\_\_
	\item \_\_ARM\_ARCH\_6\_\_
	\item \_\_ARM\_ARCH\_6J\_\_
	\item \_\_ARM\_ARCH\_6K\_\_
	\item \_\_ARM\_ARCH\_6Z\_\_
	\item \_\_ARM\_ARCH\_6ZK\_\_
	\item \_\_ARM\_ARCH\_6T2\_\_
	\item \_\_ARM\_ARCH\_7M\_\_
	\item \_\_ARM\_ARCH\_7A\_\_
\end{itemize}

\subsubsection{By Parameter Listing}

More default options can be read back from the toolchain's compiler in this way:

\begin{ptxshell}[escapechar=~]{^}
~\$~ touch test.c
~\$~ arm-linux-gcc -S --verbose-asm
\end{ptxshell}

The result in \texttt{test.s} will look like this:

\begin{ptxshell}[escapechar=~]{^}
   .arch armv4t
   .fpu softvfp
   .eabi_attribute 20, 1
   .eabi_attribute 21, 1
   .eabi_attribute 23, 3
   .eabi_attribute 24, 1
   .eabi_attribute 25, 1
   .eabi_attribute 26, 2
   .eabi_attribute 30, 6
   .eabi_attribute 18, 4
   .file "test.c"
@ GNU C (ctng-1.6.1) version 4.4.3 (arm-none-linux-gnueabi)
@  compiled by GNU C version 4.3.0 20080428 (Red Hat 4.3.0-8), GMP version 4.3.1, MPFR version 2.4.2-p2.
@ GGC heuristics: --param ggc-min-expand=98 --param ggc-min-heapsize=128213
@ options passed:  test.c -march=armv4t -mtune=arm920t -mfloat-abi=soft
@ -fverbose-asm
@ options enabled:  -falign-loops -fargument-alias -fauto-inc-dec
@ -fbranch-count-reg -fcommon -fearly-inlining
@ -feliminate-unused-debug-types -ffunction-cse -fgcse-lm -fident
@ -finline-functions-called-once -fira-share-save-slots
@ -fira-share-spill-slots -fivopts -fkeep-static-consts
@ -fleading-underscore -fmath-errno -fmerge-debug-strings
@ -fmove-loop-invariants -fpeephole -freg-struct-return -fsched-interblock
@ -fsched-spec -fsched-stalled-insns-dep -fsigned-zeros
@ -fsplit-ivs-in-unroller -ftrapping-math -ftree-cselim -ftree-loop-im
@ -ftree-loop-ivcanon -ftree-loop-optimize -ftree-parallelize-loops=
@ -ftree-reassoc -ftree-scev-cprop -ftree-switch-conversion
@ -ftree-vect-loop-version -funit-at-a-time -fverbose-asm
@ -fzero-initialized-in-bss -mglibc -mlittle-endian -msched-prolog
@ -mthumb-interwork

@ Compiler executable checksum: cd60c209f399d02b2e94fe81b4cfa8ba

   .ident   "GCC: (ctng-1.6.1) 4.4.3"
   .section .note.GNU-stack,"",%progbits
\end{ptxshell}

The important line here is the one with the \texttt{-march=armv4t}. It shows
the default setting for code generation.

\subsection{Discovering Toolchain's Library Optimization}

Beside the compiler defaults, also the basic C library coming with the toolchain
is an important part. As it comes with the toolchain in a binary data format,
the way it was compiled must also match our architecture.

\begin{ptxshell}[escapechar=~]{^}
~\$~ readelf -A <toolchain-install-directory>/arm-none-linux-gnueabi/lib/libc-2.9.so
Attribute Section: aeabi
File Attributes
  Tag_CPU_name: "4T"
  Tag_CPU_arch: v4T
  Tag_ARM_ISA_use: Yes
  Tag_THUMB_ISA_use: Thumb-1
  Tag_ABI_PCS_wchar_t: 4
  Tag_ABI_FP_denormal: Needed
  Tag_ABI_FP_exceptions: Needed
  Tag_ABI_FP_number_model: IEEE 754
  Tag_ABI_align8_needed: Yes
  Tag_ABI_enum_size: int
\end{ptxshell}

The important information is the \texttt{Tag\_CPU\_arch: v4T} line which matches
our architecture.

Note: running the command on the \texttt{libc-2.0.so} library is an example
only. You can run it on any target library you find in your toolchain.

\begin{important}
It is always possible to overwrite the architecture the compiler creates code
for by command line parameters. But it only works for newly compiled code. It
does not help for the basic C library when this component was compiled for a
different architecture. In this case there is no way to use this toolchain.
\end{important}

\subsection{BSP changes to use a foreign Toolchain}

Each \ptxdist{} based board support package is configured for a specific
toolchain vendor, a specific compiler version, toolchain name and basic C
library version. This is to ensure a user uses the same environment the release
tests where made of and to get a reliable build.

Using another toolchain than the already defined one means, all these
specified values must be changed. To do so, we run:

\begin{ptxshell}[escapechar=|]{^}
|\$| ptxdist platformconfig
 architecture --->
   toolchain  --->
      ()  check for specific toolchain vendor
      (4.4.3) check for specific gcc version
      (2.9) check for specific glibc version
      (arm-linux) gnu target
\end{ptxshell}

The settings above will switch off the vendor check and configures the board
support package for a \texttt{gcc-4.4.3} with a \texttt{glibc-2.9} and a
toolchain/compiler name of \texttt{arm-linux}.

After this change \ptxdist{} is unable to find the toolchain by its own. In
this case an additional step is required to prepare the board support package
for the build:

\begin{ptxshell}[escapechar=|]{^}
|\$| ptxdist toolchain <toolchain-install-directory>/bin
\end{ptxshell}

After that, a regular build can happen.

\begin{important}
Don't expect a successfull build after changing to a foreign toolchain. It is
impossible to provide code that builds on any compiler and compiler releases and
any C library in the wild.
\end{important}
