%
% Charset: utf8
% Content type: Plain LATEX
% Content: This chapter describes the preparations to be done
%          prior building the Mini2440 board support package.
%
% Copyright Jürgen Beisert <jbe@pengutronix.de>, 2011
%
% This work is licensed under the Creative Commons Attribution 3.0 Unported License.
% To view a copy of this license, visit:
%           http://creativecommons.org/licenses/by/3.0/
% or send a letter to:
% Creative Commons, 444 Castro Street, Suite 900, Mountain View, California, 94041, USA.
%
% Refer the file CREDITS for all people working on this document.
%
% This file content will be part of the "OSELAS.BSP-Pengutronix-Mini2440-Quickstart.pdf"
%
% Note: This document uses some externaly defined LATEX commands. If you try to
% run LATEX only on this file it will fail due to the absense of these commands
% All these new commands are starting with 'ptxdist'.
%

\section{Feature Dependend Configurations}	\label{sec:featuremini2440}

The FriendlyARM Mini2440 comes in various incarnations. Mostly they differ in
the NAND memory size, but also other features may be present or not. Read the
following sub-sections to adapt this board support package to meet exactly
your Mini2440 requirements.

Note: In this documentation the FriendlyARM Mini2440 with 64 MiB of NAND
memory is the reference platform. However, everything mentioned herein is also
valid for Mini2440s shipped with more than 64 MiB of NAND.

\subsection{Identify Your Mini2440}		\label{sec:identifymini2440}

The FriendlyARM Mini2440s are shipped with various NAND memory sizes. The
smallest is a 64 MiB unit, the largest one comes with 1 GiB of NAND memory.

As this kind of memory needs some special treatment depending on its internal
layout, we must distinguish between them prior to generating any images. This board
support package comes with two configurations:

\begin{itemize}
 \item \texttt{platformconfig-NAND-64M} for the Mini2440 with 64 MiB of NAND
   memory
  \begin{itemize}
   \item the NAND device is marked with the text \texttt{K9F1208}
  \end{itemize}
 \item \texttt{platformconfig-NAND-128M} for the Mini2440 with
  128 MiB of NAND memory or more
  \begin{itemize}
    \item these NAND devices are marked with the text \texttt{K9F1G08},
   \texttt{K9F2G08} or \texttt{K9F8G08}.
  \end{itemize}
\end{itemize}

This is important while performing the platform selection step in section
\ref{sec:selectingahardwareplatform}.
As this section references the 64 MiB NAND configuration
(\texttt{platformconfig-NAND-64M}), we must select the 128 MiB configuration
(\texttt{platformconfig-NAND-128M}) instead, if we are using a NAND memory
larger than 64 MiB.

\begin{important}
Running a 64 MiB configuration on a 128 MiB (or above) Mini2440 will
give us many confusing error messages (the same the other way around).
\end{important}

What differs in both configurations:
\begin{itemize}
 \item erase block size (16 kiB versus 128 kiB)
 \item JFFS2 root filesystem creation (needs different parameters)
 \item count of spare blocks (important for NAND memory usage)
 \item partition sizes (due to different spare block counts)
\end{itemize}

\subsection{Network Adaptions}		\label{sec:networkadaptions}

The default network configurations for the bootloader and the Linux kernel are
located in different files in this board support package. These files must be
changed in order to meet the local network requirements, to enable the
bootloader and the Linux kernel to communicate via network.

\begin{important}
The network configuration can still be changed later on when the Mini2440 is
up and running. Changing it prior the build is more for convenience.
\end{important}

\subsubsection{Bootloader Barebox}		\label{sec:bootloadernetwork}

As there is no generic network setting available, some changes to our own
network should be done prior building the board support package.

To do so, we should open one of the following files with our favourite editor:

\begin{itemize}
 \item \texttt{configs/\ptxdistPlatformName /barebox-64m-env/config} if we are using
 a Mini2440 with 64 MiB NAND
 \item \texttt{configs/\ptxdistPlatformName /barebox-128m-env/config} if we are using
 a Mini2440 with 128 MiB or larger NAND
\end{itemize}

These settings are relevant only for the bootloader. The file content will be
the default settings later on, when we are using the Mini2440. Default
settings mean they can be permanently changed at run-time later on. But, whenever
the bootloader loses its environment it will fall back to the settings in this
file. So, to avoid making more changes at run-time than required, we should do the
settings carefully here.

We can keep the \texttt{ip=dhcp} option enabled. This requires a DHCP server in
the network to be able to update the NAND memory content or to use NFS root
filesystem while developing our application. In this case at least the
\texttt{eth0.ethaddr} must be set, to give our network device an unique MAC
address.

If no DHCP server is available, a static network setting can be used instead.
We just comment out the \texttt{ip=dhcp} option and enable all the
\texttt{eth0.*} lines and give them appropriate values.

If we want to use an NFS based root filesystem, we also should adapt the
\texttt{nfsroot} setting.

\begin{important}
Don't forget the setting of an unique MAC address. At least the entry
\texttt{eth0.ethaddr} must be set.
\end{important}

\subsubsection{Linux Kernel}			\label{sec:linuxnetwork}

To define network settings used at the run-time of the Linux kernel, we must
adapt the file
\texttt{configs/platform-friendlyarm-mini2440/projectroot/etc/network/interfaces}
instead.

\subsection{Feature Adaptions}		\label{sec:featureadaptions}

Other features of the Mini2440 are:

\begin{itemize}
 \item the attached LCD
 \item if the touch facility is used
 \item if a camera is present
\end{itemize}

These features can be enabled/disabled or configured at run-time with the kernel
parameter \texttt{mini2440=}. The content of this parameter is also configured
in the \texttt{config} files mentioned above.

It is important that the LCD is configured correctly, so that it works at
run-time. Here is a list of currently known LCDs:

\begin{itemize}
 \item \textbf{0 N35}: 3.5" TFT + touchscreen (NEC NL2432HC22-23B/LCDN3502-23B)
 \item \textbf{1 A70}: 7" TFT + touchscreen (Innolux AT070TN83)
 \item \textbf{2}: VGA shield
 \item \textbf{3 t35}: 3.5" TFT + touchscreen (TD035STED4)
 \item \textbf{4}: 5.6" TFT + touchscreen (Innolux AT056TN52)
 \item \textbf{5 x35}: 3.5" TFT + touchscreen (Sony ACX502BMU-7)
 \item \textbf{6 w35i}: 3.5" TFT + touchscreen (Sharp LQ035Q1DG06)
% Is there a difference between "W35" and "W35i" display unit?
 \item \textbf{7 N43}: 4.3" TFT + touchscreen (NEC NL4827HC19-01B)
 \item \textbf{7 N43i}: 4.3" TFT + touchscreen (Sharp LQ043T3DX02)
\end{itemize}

The list above corresponds to the number (beginning with 0) given to the
\texttt{mini2440=} kernel parameter to define the LCD in use.

When starting the kernel later on, it will output the list of supported displays
with the currently selected one embraced.

\begin{ptxshell}[escapechar=|]{^}
MINI2440: LCD [0:240x320] 1:800x480 2:1024x768 3:240x320 4:640x480 5:240x320 6:320x240
\end{ptxshell}

As all of these existing LCDs differ in size and resolution, also userland may
need more information than only their resolution. If we run Qt based applications
the Qt library must know some additional data about the display as well. At
least the physical size of the visible display area is an important value, as
Qt uses this information to calculate the font's scale.

\begin{important}
The following changes are not required if you are using the 3.x kernel of
this BSP. This kernel provides the visible size of the attached screen to userland.
If you are using a different kernel, or the \textit{VGA shield} the following changes
are still to be done.
\end{important}

The BSP comes with a pre-configuration for the portrait \textit{N35}
240 x 320 display. Its visible display area size is: width 53 mm, height 71 mm.

To forward this additional information to Qt, the file
\texttt{configs/platform-friendlyarm-mini2440/projectroot/etc/profile.environment}
exists in the BSP. We can edit it prior the build and change the size settings
according to our own display if it differs from the default one. This file will
be part of the root filesystem and used at run-time.

After these changes are made, we can continue building the board support
package.
